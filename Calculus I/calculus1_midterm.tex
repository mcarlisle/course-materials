\documentclass[10pt]{article}
\setlength{\oddsidemargin}{.5in}
\setlength{\evensidemargin}{.5in}
\setlength{\textwidth}{6.0in}
\setlength{\topmargin}{0.0in}
\addtolength{\topmargin}{-1.0in}
\setlength{\footskip}{.5in}
\setlength{\textheight}{9.5in}
\linespread{1.1}

% Michael Carlisle, Baruch College, CUNY, MTH 2610, Fall 2016
% github.com/mcarlisle/lecture-notes (posted Jan 2019)

\usepackage{graphicx}
\usepackage{epsfig}
\usepackage{amssymb}
\usepackage{amsmath}

\newcommand{\prob}[1]{\vspace{5mm} \noindent \textbf{Problem #1} \,\,}

\newcommand{\header}{\begin{center}
Calculus I Midterm Exam 
\end{center}
}

\newcommand{\intro}{
This test is worth a total of 100 points (out of 110). 

Write your name on every one of the four pages of this test. 

Write all work, labeled per problem. Circle your answers on each problem. 

Work must be shown to receive credit for a problem. 

(Even if you can ``do the problem in your head'', you can write down the steps you used.) 

Calculators are not permitted.

%\vspace{5mm}
\bigskip

\noindent READ, AND PRINT AND SIGN YOUR NAME BEFORE BEGINNING THE TEST. \\
I will neither give nor receive unauthorized assistance on this exam. \\
Printed Name \underline{\hspace{60mm}} \,\,\, Signature \underline{\hspace{60mm}}
}


\begin{document}

\pagenumbering{arabic}

%---:----1----:----2----:----3----:----4----:----5----:----6----:----7----:---

\header

\intro

\vspace{20mm}

\noindent (\#1-2: 5 points each) Give the following limits. If the limit does not exist, write ``DNE''.

\prob{1} \[ \lim_{x \to 5} \frac{\sin(2x-10)}{3x-15} =  \] % simple 

\vspace{40mm}

\prob{2} \[ \lim_{x \to -1} \frac{x}{x^2 - 1} = \] % 1.4 p. 79 #44

%\prob{3} $\lim_{x \to 3} \left\{ \begin{array}{ll} -2x, & x \leq 2 \\ x^2 - 4x + 1, & x > 2 \end{array} \right.$

\pagebreak

\header 

%\vspace{20mm}

\noindent (\#3-5: 5 points each) Find the constants that make each function continuous on the entire real line, and describe why with limits. % 

\prob{3} $f(x) = \left\{ \begin{array}{ll} x^3, & x \leq 2 \\ ax^2, & x > 2 \end{array} \right.$  % 1.4 p. 80 #65

\vspace{50mm}

\prob{4} $f(x) = \left\{ \begin{array}{ll} 2, & x \leq -1 \\ ax + b, & -1 < x < 3 \\ -2, & x \geq 3 \end{array} \right.$ \\ % 1.4 p. 80 #67

\vspace{50mm}

%\pagebreak

%\header

\prob{5} $f(x) = \left\{ \begin{array}{ll} \frac{x^2 - a^2}{x - a}, & x \neq a \\ 8, & x = a \end{array} \right.$ \\ % 1.4 p. 80 #68

%\vspace{30mm}

\pagebreak

\header


\prob{6} (10 points) State the definition of the derivative of a function $f(x)$: % 4 defns @ 5 pts each
\begin{itemize}
\item[a) ] in words;
\item[b) ] with a limit.
\end{itemize}

\vspace{50mm}

\prob{7} (5 points) State the product rule for the product of the three differentiable functions $f(x)$, $g(x)$, and $h(x)$. 

\vspace{50mm}

\prob{8} (5 points) State the chain (onion) rule for the composition of differentiable functions $f(x)$ (outer) by $g(x)$ (inner). 


\pagebreak

\header

\noindent (\#9-11: 10 points each) Give $\frac{dy}{dx}$ for each function and evaluate at the given point $(x,y)$. \\ If the derivative does not exist, write ``DNE''. % 2 implicit @ 10 pts each

\prob{9} $y^2 = \frac{x^2 - 49}{x^2 + 49}$; (7,0) \\ %2.5 p. 146 #23

\vspace{40mm}

%$(x+y)^3 = x^3 + y^3$; (-1, 1) \\ %2.5 p. 146 #24

%\prob{10} $x^{2/3} + y^{2/3} = 5$ ; (8,1) \\ %2.5 p. 146 #25

%$x^3 + y^3 = 6xy - 1$ ; (2,3) \\ %2.5 p. 146 #26

%$\tan(x+y) = x$ ; (0,0) \\ %2.5 p. 146 #27

\prob{10} $x \cos y = 1$ ; $(2, \pi/3)$ \\ %2.5 p. 146 #28

\vspace{40mm}

%\noindent (10 points) Give $\frac{d^2y}{dx^2}$ in terms of $x$, $y$, and $y'$. % 1 implicit @ 10 pts 

%\vspace{10mm}

\prob{11} $x^2 y^2 - 2x = 3$; $(1, \sqrt{5})$ %2.5 p. 147 #46

\vspace{40mm}

%$1 - xy = x - y$ \\ %2.5 p. 147 #48

%$y^2 = x^3$ \\ %2.5 p. 147 #49


\prob{12} (5 points) What is the equation of the line tangent to the curve at the point given in Problem 11?


\pagebreak

\header

\prob{13} (20 points) %For each function and interval given: 
\[ f(x) = \left\{ \begin{array}{ll} -x^3 + 1, & x \leq 0 \\ -x^2 + 2x, & x > 0 \end{array} \right. \]
\begin{itemize}
\item[a) ] Describe the continuity of the function.
\item[b) ] Find the asymptotes (horizontal and vertical) of the function, if any.
\item[c) ] Find the critical numbers of the function, if any.
\item[d) ] Find the open intervals on which the function is increasing and decreasing.
\item[e) ] Find all absolute (global) and relative (local) extrema of the function.
\item[f) ] Describe the concavity of the function, and give any points of inflection.
\item[g) ] Sketch the graph of the function.
\end{itemize}

%\vspace{10mm}

%\prob{12} $f(x) = x^{1/3} + 1$ \\ % 3.3 p. 186 #27

%$f(x) = (x-3)^{1/3}$ \\ % 3.3 p. 186 #30

%\prob{13} $f(x) = \frac{2x}{x+3}$ \\ % 3.3 p. 186 #34

%$f(x) = \frac{x^2 - 2x + 1}{x + 1}$ \\ % 3.3 p. 186 #37


%\vspace{55mm}

\pagebreak 

\header

\prob{14} (10 points) Give the $y$ that minimizes the total length of three line segments: \\
one from $(-3,0)$ to $(0,y)$, one from $(3,0)$ to $(0,y)$, and one from $(0,y)$ to $(0,5)$, with $0 \leq y \leq 5$. \\
(Sketch and label a graph to understand the question.)

%(10 points) Find the maximal total area of a regular hexagon and an equilateral triangle with total perimeter 30.


\end{document}
