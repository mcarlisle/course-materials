\documentclass[10pt]{article}
%\setlength{\oddsidemargin}{.5in}
%\setlength{\evensidemargin}{.5in}
\setlength{\textwidth}{5.5in}
\setlength{\topmargin}{0.0in}
\addtolength{\topmargin}{-0.5in}
\setlength{\footskip}{.5in}
\setlength{\textheight}{9.4in}
\linespread{1.1}

% Michael Carlisle, Baruch College, CUNY, MTH 3010, Fall 2017
% github.com/mcarlisle/course-materials (posted Jan 2019)

\usepackage[dvips]{graphicx}
\usepackage{epsfig}
\usepackage{amssymb}
\usepackage{amsmath}

\newcommand{\prob}[1]{\vspace{10mm} \noindent \textbf{Problem #1} \,\,}
\newcommand{\header}{
\begin{center}
Calculus II Quiz \#1
\end{center}

\vspace{2mm}

}

\newcommand{\namefield}{
\noindent Printed Name \underline{\hspace{50mm}} \,\,\, Signature \underline{\hspace{50mm}}
}



\newcommand{\inst}[2]{
Show your work and clearly label your answers on this quiz. \emph{No scrap paper, calculators, or notes are allowed} (or needed). This quiz is scored out of #1 points. (There are #2 points possible.) You have 30 minutes to complete the quiz.

To get credit on a problem, you \emph{must} show work. Even if you can do the work in your head, the point of these exercises is to get you to articulate your thought processes.
}


\begin{document}

\pagenumbering{arabic}

%---:----1----:----2----:----3----:----4----:----5----:----6----:----7----:---

\namefield

\header

\inst{55}{65}


\prob{1} (10x3 pts) % 5.3
For each function, give a domain (perhaps restricted) on which the function is one-to-one. Give the range of the function under that domain, and the inverse function.
\begin{itemize}
\item[(a) ] $f(x) = \frac{2x + 5}{x + 3}$
\item[(b) ] $g(x) = \sqrt{ \sqrt{x} - 10}$
\item[(c) ] $h(x) = \arcsin(x - 1)$
\end{itemize}

\pagebreak

\header

\prob{2} (10 pts) % 5.3

Compute the equation of the line tangent to the \emph{inverse} of the function $f(x) = 8 - x^3$ at the point $(a, f^{-1}(a)) = (7,1)$. 

\pagebreak

\header

\prob{3} (10+5 pts) % 5.6
Compute \[ \frac{d}{dx} \left( \cos( \arcsin(\sqrt{x}) ) \right) \] and evaluate the derivative at $x = \frac{1}{4}$.
(Hint: Notice that much of this problem simplifies!)

\pagebreak

\header

\prob{4} (10 pts) % 5.7
Compute \[ \int_0^1 \frac{x}{x^4 + 4} dx. \]
Hint: 
\[ \frac{d}{dx} \left(\arctan \left(u \right) \right) = \frac{du}{u^2 + 1}. \]


\end{document}
