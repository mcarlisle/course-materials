\documentclass[10pt]{article}
%\setlength{\oddsidemargin}{.5in}
%\setlength{\evensidemargin}{.5in}
\setlength{\textwidth}{5.5in}
\setlength{\topmargin}{0.0in}
\addtolength{\topmargin}{-0.5in}
\setlength{\footskip}{.5in}
\setlength{\textheight}{9.4in}
\linespread{1.1}

% Michael Carlisle, Baruch College, CUNY, MTH 3010, Fall 2017
% github.com/mcarlisle/course-materials (posted Jan 2019)

\usepackage[dvips]{graphicx}
\usepackage{epsfig}
\usepackage{amssymb}
\usepackage{amsmath}

\newcommand{\prob}[1]{\vspace{10mm} \noindent \textbf{Problem #1} \,\,}
\newcommand{\header}{
\begin{center}
Calculus II Quiz \#5
\end{center}

\vspace{2mm}

}

\newcommand{\namefield}{
\noindent Printed Name \underline{\hspace{50mm}} \,\,\, Signature \underline{\hspace{50mm}}
}


\newcommand{\inst}[2]{
Show your work and clearly label your answers on this quiz. \emph{No scrap paper, calculators, or notes are allowed} (or needed). This quiz is scored out of #1 points. (There are #2 points possible.) You have 60 minutes to complete the quiz.

To get credit on a problem, you \emph{must} show work. Even if you can do the work in your head, the point of these exercises is to get you to articulate your thought processes.
}


\begin{document}

\pagenumbering{arabic}

%---:----1----:----2----:----3----:----4----:----5----:----6----:----7----:---

\namefield

\header

\inst{100}{120}


% 9.1
(5+3+2 pts each) For each sequence, 
\begin{itemize}
\item[(a) ] list the first 5 terms;
\item[(b) ] say if the sequence converges (and to what limit) or diverges (and how);
\item[(c) ] say if the sequence is: bounded/unbounded, strictly/monotonic increasing/decreasing.
\end{itemize}

\prob{1} $a_n = \cos(\frac{n \pi}{3})$, $n \geq 0$

\vspace{30mm}

\prob{2} $b_n = \frac{4 n^2}{n^2 + 5}$, $n \geq 1$

\vspace{30mm}

\prob{3} $c_1 = 0.1$, $c_2 = 3$, $c_{n+2} = c_n c_{n+1}$, $n \geq 1$

\vspace{30mm}

\prob{4} $d_1 = 1$, $d_2 = 4$, $d_{n+2} = d_n d_{n+1}$, $n \geq 1$

\pagebreak

\header

(5+3+2 pts each) For each series, 
\begin{itemize}
\item[(a) ] list the first 5 terms in the sequence of partial sums;
\item[(b) ] say if the series converges (and to what limit) or diverges (and how);
\item[(c) ] say if the series is: bounded/unbounded, strictly/monotonic increasing/decreasing.
\end{itemize}

\prob{5} $\sum\limits_{n = 0}^{\infty} \left( -\frac{2}{5} \right)^n$

\vspace{25mm}

\prob{6} $\sum\limits_{n = 1}^{\infty} \frac{3}{n}$

\vspace{25mm}

\prob{7} $\sum\limits_{n = 1}^{\infty} \frac{3}{2^n}$

\vspace{25mm}

\prob{8} $\sum\limits_{n = 1}^{\infty} \ln\left(\frac{n}{n+2}\right)$

\pagebreak

\header

(5+5 pts each) For each decimal, 
\begin{itemize}
\item[(a) ] rewrite the decimal as a geometric series;
\item[(b) ] rewrite the decimal as a ratio of whole numbers (in lowest terms).
\end{itemize}

\prob{9} $0.\overline{45}$

\vspace{60mm} 

\prob{10} $0.3\overline{2}$

\pagebreak

\header

(4+4+2 pts each) 
\begin{itemize}
\item[(a) ] Use the integral test to determine if the series converges or diverges. 
\item[(b) ] Use the comparison test to determine if the series converges or diverges. 
\item[(c) ] Give the limit, if the series converges; explain the divergence if the series diverges.
\end{itemize}

\vspace{10mm}

\prob{11}
$\sum\limits_{n=1}^{\infty} \frac{5}{n^2}$

\vspace{60mm}

\prob{12} 
$\sum\limits_{n=2}^{\infty} \frac{6n^2 - 5}{6n^3 - 15n + 2}$



\end{document}
