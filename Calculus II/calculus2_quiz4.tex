\documentclass[10pt]{article}
%\setlength{\oddsidemargin}{.5in}
%\setlength{\evensidemargin}{.5in}
\setlength{\textwidth}{5.5in}
\setlength{\topmargin}{0.0in}
\addtolength{\topmargin}{-0.5in}
\setlength{\footskip}{.5in}
\setlength{\textheight}{9.4in}
\linespread{1.1}

% Michael Carlisle, Baruch College, CUNY, MTH 3010, Fall 2017
% github.com/mcarlisle/course-materials (posted Jan 2019)

\usepackage[dvips]{graphicx}
\usepackage{epsfig}
\usepackage{amssymb}
\usepackage{amsmath}

\newcommand{\prob}[1]{\vspace{10mm} \noindent \textbf{Problem #1} \,\,}
\newcommand{\header}{
\begin{center}
Calculus II Quiz \#4
\end{center}

\vspace{2mm}

}

\newcommand{\namefield}{
\noindent Printed Name \underline{\hspace{50mm}} \,\,\, Signature \underline{\hspace{50mm}}
}


\newcommand{\inst}[2]{
Show your work and clearly label your answers on this quiz. \emph{No scrap paper, calculators, or notes are allowed} (or needed). This quiz is scored out of #1 points. (There are #2 points possible.) You have 30 minutes to complete the quiz.

To get credit on a problem, you \emph{must} show work. Even if you can do the work in your head, the point of these exercises is to get you to articulate your thought processes.
}


\begin{document}

\pagenumbering{arabic}

%---:----1----:----2----:----3----:----4----:----5----:----6----:----7----:---

\namefield

\header

\inst{50}{60}


\prob{1} (10 pts) % 8.5

\[ \int \frac{x+25}{x^2 + 24x + 148} dx = \]

\vspace{60mm} 

%\prob{2} (10 pts)

%\[ \int \frac{x}{84x^4 - 1} dx = \]


\pagebreak

\header

\prob{2} (10+5 pts) % 4.6

Use the Trapezoidal Rule, with $n=6$, to approximate $\int_0^3 (x^2 + 4) dx$, and compare to the exact answer of the integral.

(Hint: $1/2$ on the outside, and 1, 2, 2, ..., 2, 1 on the inside.)
\pagebreak

\header

\prob{3} (5+5+5 pts) % 8.7

\begin{itemize}
\item[(a) ] $\lim\limits_{x \to 5} \frac{e^{5-x}}{(x-5)^2} = $
\item[(b) ] $\lim\limits_{x \to 0} \frac{\sqrt{25-x^2} - 5}{x^2} = $
\item[(c) ] $\lim\limits_{x \to \frac{\pi}{2}} \frac{\cos(5x)}{\cos(3x)} = $
\end{itemize}

\pagebreak

\header

\prob{4} (10+10 pts) % 8.8

\begin{itemize}
\item[(a) ] Find the area of the region bounded by $x=0$, $y=0$, and $y = x^2 e^{-x}$; that is, compute 
\[ \int_0^{\infty} x^2 e^{-x} dx = \]
\item[(b) ] Compute the volume of the solid formed by revolving the curve $y = e^{-x}$, starting at $x=0$ and going right (limit as $x \to \infty$) around the $x$-axis.

\vspace{5mm}

(Conclusion: there are some abstract shapes that have infinite length and finite area, and abstract solids that have infinite surface area but contain finite volume. \\
This phenomenon is referred to as \emph{Gabriel's horn} or \emph{Torricelli's trumpet}.)
\end{itemize}



\end{document}
